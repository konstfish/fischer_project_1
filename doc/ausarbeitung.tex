\documentclass[12pt, letterpaper]{article}
\usepackage[utf8]{inputenc}
\usepackage{cite}
\usepackage{float}
\usepackage{tikz}

\graphicspath{{images/}}

\title{Simulation einer verteilten Synchronisation mit einem zentralen Koordinator}
\author{David Fischer, 5CHIF}
\date{Dezember 2020}

\begin{document}

\begin{titlepage}
\maketitle
\end{titlepage}

\tableofcontents
\newpage

\section{Einleitung}

\subsection{Problematik}
Laut Angabe war das Ziel dieser Aufgabe eine Simulation einer verteilten Synchronisation mit einem zentalen Koordinator zu erstellen. Die Simulation soll mit einer beim Aufruf definierten Anzahl an Nodes gestartet werden.

\begin{figure}[H]
    \centering
    \resizebox{0.5\textwidth}{!}{\input{images/situation1.pdf_tex}}
    \caption{1}
    \label{fig:situation1}
\end{figure}

\begin{figure}[H]
    \centering
    \resizebox{0.5\textwidth}{!}{\input{images/situation2.pdf_tex}}
    \caption{2}
    \label{fig:situation2}
\end{figure}


\begin{figure}[H]
    \centering
    \resizebox{0.5\textwidth}{!}{\input{images/situation3.pdf_tex}}
    \caption{3}
    \label{fig:situation3}
\end{figure}


\subsubsection{Gegenseitiger Ausschluss (Mutual Exclusion)}

\subsubsection{Verteilte Synchronisation}

\section{Implementierung}

\subsection{Externe Bibliotheken}

\section{Verwendung}

\subsection{Kommandozeilen Argumente}

\section{Projekt Struktur}


% .bib include & references
\newpage
\bibliography{references}
\bibliographystyle{plain}
\end{document}